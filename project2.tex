\documentclass[a4paper,12pt, english]{article}
\usepackage[T1]{fontenc}
\usepackage[utf8]{inputenc}
\usepackage{graphicx}
\usepackage{babel}
\usepackage{amsmath}
\usepackage{ulem}
\usepackage{a4wide}
\usepackage{graphicx}
\usepackage{listings}
\usepackage{tabularx}
\usepackage{tabulary}


\begin{document}
\section*{Project 2}
\subsubsection*{Scr\"odinger's equation for two electrons in a three-dimensional harmonic oscillator well }

Our task in this project is to solve Scr\"odinger's equation for two electrons in a three-dimensional oscillator well without the repulsive Coulomb interaction. We are going to solve this equation by reformulating it in a discretized form as an eigenvalue equation to be solved with Jacobi's method. 
\\
The radial part of the wave function, $R(r)$, is a solution to 
$$ -\frac{\hbar^2}{2m}(\frac{1}{r^2}\frac{d}{dr}r^2\frac{d}{dr} - \frac{l(l+1)}{r^2})R(r) + V(r)R(r) = ER(r)  $$
\\
In our case V(r) is the harmonic oscillator potential $\frac{1}{1}kr^2$ with $k=m\omega^2$. $E$ is the energy of the harmonic oscillator in three dimensions. The oscillator frequency is $\omega$ and the energies are
$$ {E_n}_l = \hbar\omega(2n + l + \frac{3}{2}) $$
with $n = 0,1,2,...$ and $l = 0,1,2,...$  \\
When we substitute $R(r) = \frac{1}{r}u(r)$ with boundary conditions $u(0) = 0$ and $u(\infty) = 0$ and introduce dimensionless variable $\rho = \frac{1}{\alpha}r$ we can rewrite Scr\"odinger's equation as 
$$ - \frac{d^2}{d\rho^2}u(\rho) + \rho^2u(\rho) = \lambda u(\rho) $$
with $\alpha = \frac{\hbar^2}{mk}^1/4$ and $\lambda = \frac{2m\alpha^2}{\hbar^2}E$. \\
Our goal is to rewrite this equation as a matrix eigenvalue problem. We use the standard expression for the second derivative $$ u'' = \frac{u(\rho + h) -2u(\rho) + u(\rho -h)}{h^2} + O(h^2) $$
where $h$ is our step.
With a given number of steps, $n_{steps}$, we define the step $h$ as $$h = \frac{\rho_{max}-\rho_{min}}{n_{step}}. $$
We define an arbitrary value of $\rho$ to be $\rho_i = \rho_{min} + ih$. \\
.....\\

With these definitions the Scr\"odinger equation takes the following form $$ d_iu_i + e_{i-1}u_{i-1} + e_{i+1}u_{i+1} = \lambda u_i $$
where $u_i$ is the unknown. We can write this equation as a matrix eigenvalue problem \\
----
\\


We define the quantities $tan(\theta) = t = \frac{s}{c}$, with $s=sin(\theta)$ and $c = cos(\theta)$ and 

$$cot(2\theta) = \tau = \frac{a_{ll}- a_{kk}}{2a_{kl}} $$ 
Using $$cot(2\theta) = \frac{1}{2}(cot(\theta) - tan(\theta))$$ we get 
$$\tau = \frac{1}{2}(\frac{1}{t} -t) $$
$$2\tau = \frac{1}{t} - t \hspace{5mm} \mid *t $$
$$t^2 + 2\tau t - 1 = 0 \Rightarrow t = -\tau \pm \sqrt{1+\tau^2} $$


Why we should choose $t$ to be the smaller of the roots:
When doing the Jacobi transformation, we pick out the greatest non-diagonal element and set it equal to zero.
(side 216)
When changing this element we involuntary change the other elements on the same column and row. Ideally these elements would stay zero. We want the change in these elements to be as small as possible. 

Largest element:
$$ {b_k}_l = ({a_k}_k - {a_l}_l) cos(\theta) + {a_k}_l(cos(\theta)^2 - sin(\theta)^2$$

Change in corresponding row and column:
$$ {b_i}_k = {a_i}_kcos(\theta) - {a_i}_lsin(\theta) $$
$$ {b_i}_l = {a_i}_lcos(\theta) + {a_i}_ksin(\theta) $$
  
We see that the we get the lowest change when $cos(\theta) \rightarrow 1$ and $sin(\theta \rightarrow 0$.

Our goal is therefore to make $ \mid tan(\theta) \mid  = \mid \frac{sin(\theta)}{cos(\theta)} \mid$ as small as possible, and thereby make the change of the elements effected as little as possible.
To do this we choose the smallest of the roots $tan(\theta) = t = - \tau \pm \sqrt{1+\tau^2}$

For $\tau > 0$ the smallest $t$ is given by $ t = - \tau + \sqrt{1+ \tau^2}$.

For $\tau < 0$ the root $ t = - \tau - \sqrt{1+ \tau^2}$ gives the smallest change.  
  
To avoid loss of numerical precision we rewrite
$$\tau>0:$$
$$(-\tau + \sqrt{1 + \tau^2}) * \frac{\tau + \sqrt{1+\tau^2}}{\tau + \sqrt{1+\tau^2}} = \frac{1}{\tau + \sqrt{1+\tau^2}} $$
$$\tau<0:$$
$$(-\tau - \sqrt{1 + \tau^2}) * \frac{\tau - \sqrt{1+\tau^2}}{\tau + \sqrt{1-\tau^2}} = \frac{-1}{\-tau + \sqrt{1+\tau^2}}$$

By this we see that the largest value we will get for $\mid tan(\theta) \mid = 1$. And thus we know that the largest value for the absolute value of the angle must be $\frac{\pi}{4}$. \\ 
  
  
In my algorithm I make a function $off-diagonal$ that takes as argument the matrix A, the index variables k and l, and the matrix dimension n. Both A and the index variables is given by address. In this function I run over the non-diagonal elements to find the biggest value of these. When having done so, I save the index for this value in the variables k and l. The function returns the maximum value. \\ 

As long as the maximum value is bigger than my chosen error limit, $\epsilon = 1*10^-8$, I run my $rotation$ function. This function performs the Jacobi rotation. The biggest off-diagonal value is set to zero, and the belonging row and column is affected by the rotation. \\

I plead satisfied when my biggest off-diagonal matrix element is smaller than $\epsilon$. I then have my eigenvalues on the diagonal of the matrix A. \\

The eigenvalues is not ordered after value. So to find my three lowest eigenvalues I make an armadillo vector with the eigenvalues, and use the functionality sort to get them into order. I then print the first three values of the matrix. \\

We need about $n_{step} = 196$ to get the lowest three eigenvalues with four leading digits. At this step length we run our program $63376$ times to reach the given precision (if we decrease our epsilon our program runs more times). Behaviour as a function of the dimensionality of the matrix: goes as $n^2$. 

For $n=196$: \\
tqli: $time = 0.1132 s$ \\
our: $time = 16.3448 s $ \\
We are sloooow - you are faast (wanna run against me??)
 
Subproblem c:
For $n=196$
KKKARI 
 \begin{center}
 \begin{tabular}{l |  l l l}
$\omega_r$ &  The three lowest eigenvalues & & \\
 0.01 & 0.847 & 2.196 & 4.2768   \\
 0.5 & 2.231 & 4.171 & 6.402 \\
1 & 4.057 & 7.908 & 11.817\\
5 & 17.443 & 37.047 & 56.793
\end{tabular}
\end{center}

BREMNES???? kari ....
We checked it against a closed form solution. Turned off the coulomb interaction. $\omega_r = 5$
Eigenvalues: $15, 35, $
Small contribution from $ 1/ \rho)$



Plotting the eigenvectors corresponding to the three lowest eigenvalues.\\

\includegraphics[scale=0.5]{omega1rhomax5}
\includegraphics[scale=0.5]{omega1rhomax15}

Our choice of $\rho_{max}$ depends on the width of the wavepacket. If we have a wide wavepacket we must choose a large $\rho$ to get all the relevant data. We see that the wavepacket reduces quite quickly to zero, and thus we do not need a too large $\rho$.  

The width of the wavepacket is dependent on the frequency, potential $V = m \omega ^ 2$. The frequency decides the width of the curve, and thus influences our choice of $\rho$. \\ 

When you increase omega the potential increases strongly - and thus it will be much less likely that the particles will be located far from each other.  \\

\includegraphics[scale=0.5]{omega5rhomax5}
\includegraphics[scale=0.5]{omega001rhomax5}
\includegraphics[scale=0.5]{omega001rhomax1}

\end{document}
