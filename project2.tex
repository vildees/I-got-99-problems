\documentclass[a4paper,12pt, english]{article}
\usepackage[T1]{fontenc}
\usepackage[utf8]{inputenc}
\usepackage{graphicx}
\usepackage{babel}
\usepackage{amsmath}
\usepackage{ulem}
\usepackage{a4wide}
\usepackage{graphicx}
\usepackage{listings}
\usepackage{tabularx}
\usepackage{tabulary}


\begin{document}
\section*{Project 2}
\subsubsection*{Scr\"odinger's equation for two electrons in a three-dimensional harmonic oscillator well }

Our task in this project is to solve Scr\"odinger's equation for two electrons in a three-dimensional oscillator well without the repulsive Coulomb interaction. We are going to solve this equation by reformulating it in a discretized form as an eigenvalue equation to be solved with Jacobi's method. 
\\
The radial part of the wave function, $R(r)$, is a solution to 
$$ -\frac{\hbar^2}{2m}(\frac{1}{r^2}\frac{d}{dr}r^2\frac{d}{dr} - \frac{l(l+1)}{r^2})R(r) + V(r)R(r) = ER(r)  $$
\\
In our case V(r) is the harmonic oscillator potential $\frac{1}{1}kr^2$ with $k=m\omega^2$. $E$ is the energy of the harmonic oscillator in three dimensions. The oscillator frequency is $\omega$ and the energies are
$$ {E_n}_l = \hbar\omega(2n + l + \frac{3}{2}) $$
with $n = 0,1,2,...$ and $l = 0,1,2,...$  \\
When we substitute $R(r) = \frac{1}{r}u(r)$ with boundary conditions $u(0) = 0$ and $u(\infty) = 0$ and introduce dimensionless variable $\rho = \frac{1}{\alpha}r$ we can rewrite Scr\"odinger's equation as 
$$ - \frac{d^2}{d\rho^2}u(\rho) + \rho^2u(\rho) = \lambda u(\rho) $$
with $\alpha = \frac{\hbar^2}{mk}^1/4$ and $\lambda = \frac{2m\alpha^2}{\hbar^2}E$. \\
Our goal is to rewrite this equation as a matrix eigenvalue problem. We use the standard expression for the second derivative $$ u'' = \frac{u(\rho + h) -2u(\rho) + u(\rho -h)}{h^2} + O(h^2) $$
where $h$ is our step.
With a given number of steps, $n_{steps}$, we define the step $h$ as $$h = \frac{\rho_{max}-\rho_{min}}{n_{step}}. $$
We define an arbitrary value of $\rho$ to be $\rho_i = \rho_{min} + ih$. \\
.....\\

With these definitions the Scr\"odinger equation takes the following form $$ d_iu_i + e_{i-1}u_{i-1} + e_{i+1}u_{i+1} = \lambda u_i $$
where $u_i$ is the unknown. We can write this equation as a matrix eigenvalue problem \\
----
\\


We define the quantities $tan(\theta) = t = \frac{s}{c}$, with $s=sin(\theta)$ and $c = cos(\theta)$ and 

$$cot(2\theta) = \tau = \frac{a_{ll}- a_{kk}}{2a_{kl}} $$ 
Using $$cot(2\theta) = \frac{1}{2}(cot(\theta) - tan(\theta))$$ we get 
$$\tau = \frac{1}{2}(\frac{1}{t} -t) $$
$$2\tau = \frac{1}{t} - t \hspace{5mm} \mid *t $$
$$t^2 + 2\tau t - 1 = 0 \Rightarrow t = -\tau \pm \sqrt{1+\tau^2} $$


Why we should choose $t$ to be the smaller of the roots:
When doing the Jacobi transformation, we pick out the greatest non-diagonal element and set it equal to zero.
(side 216)
When changing this element we involuntary change the other elements on the same column and row. Ideally these elements would stay zero. We want the change in these elements to be as small as possible. 

Largest element:
$$ {b_k}_l = ({a_k}_k - {a_l}_l) cos(\theta) + {a_k}_l(cos(\theta)^2 - sin(\theta)^2$$

Change in corresponding row and column:
$$ {b_i}_k = {a_i}_kcos(\theta) - {a_i}_lsin(\theta) $$
$$ {b_i}_l = {a_i}_lcos(\theta) + {a_i}_ksin(\theta) $$
  
We see that the we get the lowest change when $cos(\theta) \rightarrow 1$ and $sin(\theta \rightarrow 0$.

Our goal is therefore to make $ \mid tan(\theta) \mid  = \mid \frac{sin(\theta)}{cos(\theta)} \mid$ as small as possible, and thereby make the change of the elements effected as little as possible.
To do this we choose the smallest of the roots $tan(\theta) = t = - \tau \pm \sqrt{1+\tau^2}$

For $\tau > 0$ the smallest $t$ is given by $ t = - \tau + \sqrt{1+ \tau^2}$.

For $\tau < 0$ the root $ t = - \tau - \sqrt{1+ \tau^2}$ gives the smallest change.  
  
To avoid loss of numerical precision we rewrite
$$\tau>0:$$
$$(-\tau + \sqrt{1 + \tau^2}) * \frac{\tau + \sqrt{1+\tau^2}}{\tau + \sqrt{1+\tau^2}} = \frac{1}{\tau + \sqrt{1+\tau^2}} $$
$$\tau<0:$$
$$(-\tau - \sqrt{1 + \tau^2}) * \frac{\tau - \sqrt{1+\tau^2}}{\tau + \sqrt{1-\tau^2}} = \frac{-1}{\-tau + \sqrt{1+\tau^2}}$$

By this we see that the largest value we will get for $\mid tan(\theta) \mid = 1$. And thus we know that the largest value for the absolute value of the angle must be $\frac{\pi}{4}$. \\ 
  
\subsubsection{Method}
In our algorithm we make a function $off-diagonal$ that takes as argument the matrix A, the index variables k and l, and the matrix dimension n. Both A and the index variables are given by address. In this function we run over the non-diagonal elements to find the biggest value of these. When having done so, we save the index for this value in the variables k and l. The function returns the maximum value. 

As long as the maximum value is bigger than our chosen error limit, $\epsilon = 1*10^-8$, we run our $rotation$ function. This function performs the Jacobi rotation. The biggest off-diagonal value is set to zero and, as an effect of the rotation, the belonging row and column is affected as well. Our goal is to do the rotations so that all the non-diagonal elements is zero. We plead satisfied when our biggest off-diagonal matrix element is smaller than our chosen limit $\epsilon$. We then have our eigenvalues on the diagonal of the matrix.
 
The eigenvalues are not ordered after value. So to find our three lowest eigenvalues we make a vector with the eigenvalues, and use the functionality $sort$ to get them into order. We then print the first three values of the matrix.
 
\subsubsection{Results}
In three dimensions the eigenvalues for $l=0$ are $\lambda_0 = 3, \lambda_1 = 7, \lambda_2 = 11,...$ We want to find out how many points $n_{step}$ that is needed to get the three lowest eigenvalues with four leading digits. When running our program we see that it is first for $n_{step} = 196$ we get the wanted precision. 
 
With our choice of tolerance ($\epsilon = 1*10^{-8}$) our program runs $63376$ times to reach the given precision. We see that if we decrease our $\epsilon$, the rotation must be performed more times to achieve the same precision. 

Running for different $n-values$ we see that the number of Jacobi rotations needed increases as $n$ increases.

\begin{center}
\begin{tabular}{l |  l }
n & counter \\
10 & 112   \\
20 & 555 \\
30 & 1322 \\
100 & 16178 \\
200 & 66029 \\
\end{tabular}
\end{center}  

We see that, roughly estimated, the number of transformations needed as function of the dimensionality of the matrix goes as $n^2$. \\


To check our results we used the armadillo function $eig\_sym$, that gives us both the eigenvalues and the eigenvector of our symmetric matrix. We find that the three lowest eigenvalues are $2.999, 6.999$ and $10.999$, which coincides with the results we found from our Jacobi method. By comparing to the analytical values for the eigenvalues, we see that we are within good precision. We see that the values have four leading digit, which we needed a $n=196$-matrix and $63376$ to reach in our Jacobi rotation program. However, the armadillo program runs much faster than our Jacobi solver. We found that for a $n=196$ matrix the armadillo program needed $0.0349 s$ to do the calculations, while our program used $16.3448 s$. We see that for small matrices the precision is low when using the Jacobi method, and for large matrices the time needed to reach a good precision is high. \\


To study two electrons in a harmonic oscillator well which also interact via a repulsive Coulomb interaction, we used our code from subproblem (a), but changed the potential to $$V = \omega_r^2 \rho^2 + \frac{1}{\rho} $$

BREMNES???? kari ....
We checked it against a closed form solution. Turned off the coulomb interaction. $\omega_r = 5$
Eigenvalues: $15, 35, $
Small contribution from $ 1/ \rho)$


When running for $n=196$ and $rho_max = 5$, as before, we found the lowest three eigenvalues for different $\omega$

 
\begin{center}
\begin{tabular}{l |  l l l}
$\omega_r$ & The three lowest eigenvalues \\
0.01 & 0.847 & 2.196 & 4.2768   \\
0.5 & 2.231 & 4.171 & 6.402 \\
1 & 4.057 & 7.908 & 11.817\\
5 & 17.443 & 37.047 & 56.793
\end{tabular}
\end{center}




To find the eigenvalues and eigenfunctions of the two-electron wavefunction we used, as for the one-electron case, the armadillo function $eig_sym$. We passed the eigenvectors that corresponds to the three lowest eigenvalues to a file $data.dat$ and made a python script that read out and plotted these vectors against its index. To make it easy to change variables we compiled, ran and gave the values of $n$, $\omega$ and $\rho$ from within our python script.    


Plotting the eigenvectors corresponding to the three lowest eigenvalues.\\

\includegraphics[scale=0.5]{omega1rhomax5}
\includegraphics[scale=0.5]{omega1rhomax15}

Our choice of $\rho_{max}$ depends on the width of the wavepacket. If we have a wide wavepacket we must choose a large $\rho$ to get all the relevant data. We see that the wavepacket reduces quite quickly to zero, and thus we do not need a too large $\rho$.  

The width of the wavepacket is dependent on the frequency, potential $V = m \omega ^ 2$. The frequency decides the width of the curve, and thus influences our choice of $\rho$. \\ 

When you increase omega the potential increases strongly - and thus it will be much less likely that the particles will be located far from each other.  \\

\includegraphics[scale=0.5]{omega5rhomax5}
\includegraphics[scale=0.5]{omega001rhomax5}
\includegraphics[scale=0.5]{omega001rhomax1}

\end{document}